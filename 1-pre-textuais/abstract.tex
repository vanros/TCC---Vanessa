The use of Field Programmable Gate Array (FPGAs) has been widely diffused in the market by allowing the acceleration of workloads that require a high processing power, offering the advantage of a higher performance per watt, in several scenarios. However, the alignment of what is taught at university with market reality is often ineffective due to the high cost of purchasing physical material such as circuit boards. as well as the licenses of the software tools needed to implement solutions for hardware acceleration. In this context, this final paper describes a study on the use of the Elastic Compute Cloud (EC2) F1 cloud computing instances in the teaching of of Reconfigurable Digital Electronic Systems to undergraduate students in Computer Engineering of the Department of Teleinformatics Engineering of Federal University of Ceara. These instances are equipped with high-end FPGAs and offer a pre-defined development environment, providing low development costs as \$ 1.65 per hour for FPGA applications. This study addresses the development and application of laboratory practices using this service available on Amazon Web Services. The results obtained took into account the students' success in the practice and the feedback obtained from SEEQ questionnaires that were applied in order to evaluate the effectiveness of teaching FPGAs through the practices.

% Separe as Keywords por ponto
\keywords{FPGA, Vivado, SDAccel, Cloud, Xilinx}