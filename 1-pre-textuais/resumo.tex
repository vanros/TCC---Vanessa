O uso de FPGAs (Field Programmable Gate Array) tem sido amplamente difundido no mercado, por permitirem  a aceleração de cargas de trabalho que exigem um alto poder de processamento, oferecendo a vantagem de um maior desempenho por watt em vários cenários. Contudo, o alinhamento do que é ensinado na universidade com a realidade do mercado, muitas vezes não é efetivo, devido ao alto custo da compra de material físico, como placas, e das licenças das ferramentas necessárias para a implementação de soluções para aceleração de hardware. Nesse contexto, este trabalho descreve um estudo sobre a utilização das instâncias de computação em nuvem Elastic Compute Cloud (EC2) F1, no ensino da disciplina de Sistemas Eletrônicos Digitais Reconfiguráveis ministrada aos estudantes de graduação em Engenharia de Computação do Departamento de Engenharia de Teleinformática da Universidade Federal do Ceará. Essas instâncias são equipadas com FPGAs \textit{high-end} e oferecem um ambiente de desenvolvimento pré-definido, proporcionando baixo custo de desenvolvimento, da ordem de \$1.65 por hora, de aplicações para FPGAs. Este estudo aborda elaboração, testes e aplicação de práticas de laboratório utilizando esse serviço disponível na Amazon Web Services. Os resultados obtidos levaram em consideração o sucesso dos alunos na realização das práticas e o \textit{feedback} obtido de questionários SEEQ que foram aplicados a fim de avaliar a eficácia do ensino de FPGAs por meio das práticas.


% Separe as palavras-chave por ponto
\palavraschave{FPGA, Nuvem. Xilinx}