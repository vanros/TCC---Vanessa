

Ao Prof. Dr. Ricardo Silva Thé Pontes por me orientar em minha tese de doutorado.

Ao Prof. Dr. Tobias Rafael Fernandes Neto, coordenador do Laboratório de Sistemas Motrizes (LAMOTRIZ) onde este \textit{template} foi desenvolvido. 

Ao Doutorando em Engenharia Elétrica, Ednardo Moreira Rodrigues, e seu assistente, Alan Batista de Oliveira, aluno de graduação em Engenharia Elétrica, pela adequação do \textit{template} utilizado neste trabalho para que o mesmo ficasse de acordo com as normas da biblioteca da Universidade Federal do Ceará (UFC). %AVISO: Você pode usar este template uma vez que der os devidos créditos. Portanto, mantenha este parágrafo de agradecimento.

Aos bibliotecários da Universidade Federal do Ceará: Eliene Maria Vieira de Moura, Francisco Edvander Pires Santos, Izabel Lima dos Santos, Juliana Soares Lima, Kalline Yasmin Soares Feitosa pela revisão e discussão da formatação utilizada neste \textit{template}. 

Ao aluno Thiago Nascimento do curso de ciência da computação da Universidade Estadual do Ceará que elaborou o \textit{template} do qual este trabalho foi adaptado para Universidade Federal do Ceará.

Ao Prof. Dr. Humberto de Andrade Carmona do Curso de Física da UFC pelo primeiro incentivo para o uso do \textit{Latex}.

Ao aluno de graduação em engenharia elétrica e amigo, Lohan Costa por me apresentar a plataforma \textit{ShareLatex}. 

Aos amigos de laboratório, Felipe Bandeira, Renan Barroso e Roney Coelho, pelas discussões sobre os recursos do \textit{Latex}.

Aos meus pais, irmãos e sobrinhos, que nos momentos de minha ausência dedicados ao estudo superior, sempre fizeram entender que o futuro é feito a partir da constante dedicação no presente!

Agradeço a todos os professores por me proporcionar o conhecimento não apenas racional, mas a manifestação do caráter e afetividade da educação no processo de formação profissional, por tanto que se dedicaram a mim, não somente por terem me ensinado, mas por terem me feito aprender. 

E à Fundação Cearense de Apoio ao Desenvolvimento (Funcap), na pessoa do Presidente Tarcísio Haroldo Cavalcante Pequeno pelo financiamento da pesquisa de doutorado via bolsa de estudos.
